\section{Overview}
For the CAD drawings, there are plenty of softwares available there in the market including proprietary softwares such AutoCAD, STAAD.Pro and open-source ones like LibreCAD, OpenSCAD, FreeCAD etc. They are high-end softwares that require Designing knowledge and some computer usage knowledge. Hence, a layman that needs a small work to be done, will have to learn its usage effectively and some extra efforts for doing that. Hence this project will let them use it like they are making a wish for something to be done without any efforts.


\section{Problems}
It is difficult to operate high-end CAD software for a simple layman as he does not have  much knowledge of the computer system because they are not connected to the internet in any ways. Giving training to these peoples is not easy as there is not so much manpower available to provide training and it will be expensive also. By not using the computer softwares instead of traditional drawing methods, Resources are being wasted 
on manually making the drawings at various levels.
Also it is very difficult to make the drawing design using the traditional methods these days because of the evolving computer CAD softwares.

\section{Solution Description}
The layman is given a hand-held terminal working on latest technologies. Whenever a layman or any person related to Civil Engineering need to make a drawing of any building or other entity, then he/she is asked to give the entity or the drawing related details. On the basis of these details, the software will be able to draw the design of the specified architecture.  The software prints the specified drawing and the drawing file is opened in the LibreCad.
Moreover, this system will also enable the layman to know how the system works and the how the output is generated using the software and how the drawing is made.

\section{Key Features}
\begin{itemize}
\item Parse the input entity values into the system and produces a relavent output file with entity name and parameters.
\item Export the output file in dxf format which can be imported into the CAD softwares like LibreCad.
\item Enables simplification of drawing work for the civil engineers and the layman which otherwise may end up taking more time using the traditional drawing methods. 
\end{itemize}
\section{Benefits}
\begin{itemize}
\item It enables the user to easily draw the two dimensional figures of supported entities  in a very less time.
\item Reduces the time required for the user to draw the designs of walls, circles and many other entities therefore helps in increasing efficiency.
\item It has a very simple operating method which enables even a layman to use this system and draw the required designs using this software.
\item Eliminates the manual operations used for drawings by automating the entire process.
\item It reduces the time required to do a particular drawing task and thus leads to better efficiency or productivity.
\item Enables the end user to create the drawings within few minutes.
\item Provides an interface which can be used even by the peoples who are not so well proficint in computers.
\end{itemize} 

