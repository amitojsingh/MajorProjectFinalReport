\begin{figure}[h]
\centering \includegraphics[scale=0.27]{images/github.jpg}
\caption{Github logo}
\end{figure}
\noindent GitHub is a Git repository web-based hosting service which offers all of the functionality of Git as well as adding many of its own features. Unlike Git which is strictly a command-line tool, Github provides a web-based graphical interface and desktop as well as mobile integration. It also provides access control and several collaboration features such as wikis, task management, and bug tracking and feature requests for every project.\\

\noindent GitHub offers both paid plans for private repto handle everything from small to very large projects with speed and efficiency. ositories, and free accounts, which are usually used to host open source software projects. As of 2014, Github reports having over 3.4 million users, making it the largest code host in the world.\\

\noindent GitHub has become such a staple amongst the open-source development community that many developers have begun considering it a replacement for a conventional resume and some employers require applications to provide a link to and have an active contributing GitHub account in order to qualify for a job.\\\\

\section{What is Git?}
\begin{figure}[h]
\centering \includegraphics[scale=0.3]{images/git.jpg}
\caption{Git logo}
\end{figure}
\noindent Git is a distributed revision control and source code management (SCM) system with an emphasis on speed, data integrity, and support for distributed, non-linear workflows. Git was initially designed and developed by Linus Torvalds for Linux kernel development in 2005, and has since become the most widely adopted version control system for software development.\\

\noindent As with most other distributed revision control systems, and unlike most client–server systems, every Git working directory is a full-fledged repository with complete history and full version-tracking capabilities, independent of network access or a central server. Like the Linux kernel, Git is free and open source software distributed under the terms of the GNU General Public License version 2 to handle everything from small to very large projects with speed and efficiency.\\

\noindent Git is easy to learn and has a tiny footprint with lightning fast performance. It outclasses SCM tools like Subversion, CVS, Perforce, and ClearCase with features like cheap local branching, convenient staging areas, and multiple workflows.\\

\subsection{Installation of Git}

Installation of git is a very easy process.
The current git version is: 2.0.4.
Type the commands in the terminal:\\\\
\emph{
\$ sudo apt-get update\\\\
\$ sudo apt-get install git\\\\}
This will install the git on your pc or laptop.

\subsection{Various Git Commands}

Git is the open source distributed version control system that facilitates GitHub activities on your laptop or desktop. The commonly used Git command line instructions are:-\\

\subsection*{Create Repositories}
\addcontentsline{toc}{subsection}{Create Repositories}
Start a new repository or obtain from an exiting URL

\begin{description}

\item [\$ git init [ project-name]]\\
Creates a new local repository with the specified name
\item [\$ git clone [url]]\\
Downloads a project and its entire version history

\end{description}

\subsection*{Make Changes}
\addcontentsline{toc}{subsection}{Make Changes}
Review edits and craft a commit transaction

\begin{description}

\item [\$ git status] \leavevmode \\
Lists all new or modified files to be committed

\item [\$ git diff] \leavevmode \\
Shows file differences not yet staged

\item [\$ git add [file]]\\
Snapshots the file in preparation for versioning

\item [\$ git reset [file]]\\
Unstages the file, but preserve its contents

\item [\$ git commit -m "[descriptive message]"]\\
Records file snapshots permanently in version history

\end{description}

\subsection*{Group Changes}
\addcontentsline{toc}{subsection}{Group Changes}
Name a series of commits and combine completed efforts

\begin{description}

\item [\$ git branch] \leavevmode \\
Lists all local branches in the current repository

\item [\$ git branch [branch-name]]\\
Creates a new branch

\item [\$ git checkout [branch-name]]\\
Switches to the specified branch and updates the working directory

\item [\$ git merge [branch]]\\
Combines the specified branch’s history into the current branch

\item [\$ git branch -d [branch-name]]\\
Deletes the specified branch

\end{description}


