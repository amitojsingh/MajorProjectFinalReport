\section{Feasibility Analysis}
Feasibility analysis aims to uncover the strengths and weaknesses of 
a project. In its simplest term, the two criteria to judge feasibility 
are cost required and value to be attained. As such, a well-designed 
feasibility analysis should provide a historical background of the 
project, description of the project or service, details of the 
operations and management and legal requirements. Generally, feasibility 
analysis precedes technical development and project implementation. 
There is some feasibility factors by which we can determine that 
project is feasible or not:
\begin{itemize}
\item {\bf{Technical feasibility}}: Technological feasibility is carried 
out to determine whether the project has the capability, in terms of 
software, hardware, personnel to handle and fulfill the user 
requirements. The assessment is based on an outline design of system 
requirements in terms of Input, Processes, Output and Procedures. Elearning system is technically feasible as it is built up in Open 
Source Environment and thus it can be run on any Open Source platform.
\item {\bf{Economic feasibility}}: Economic analysis is the most 
frequently used method to determine the cost/benefit factor for 
evaluating the effectiveness of a new system. In this analysis we 
determine whether the benefit is gained according to the cost invested 
to develop the project or not. If benefits outweigh costs, only then 
the decision is made to design and implement the system. It is 
important to identify cost and benefit factors, which can be categorized 
as follows:
\begin{enumerate}
\item Development costs.
\item Operating costs.
\end{enumerate}
Elearning System is also Economically feasible with 0 Development 
and Operating Charges as it is developed using open source technologies and the software is operated on Open 
Source platform.
\item {\bf {Legal feasibility}}: In this type of feasibility study, we 
basically determine whether the project conflicts with legal 
requirements, e.g. a data processing system must comply with the local 
Data Protection Acts. But Elearning System has been developed with properly Licensed technologies. 
Thus is the legal process.
\item {\bf{Operational feasibility}}: Operational feasibility is a measure 
of how well a project solves the problems, and takes advantage of the 
opportunities identified during scope definition and how it satisfies 
the requirements identified in the requirements analysis phase of system 
development. All the operations performed in the system are very quick 
and satisfy all the requirements.
\item {\bf{Behaviour Feasibility}}: In this feasibility, we check about the 
behavior of the proposed system software i.e. whether the proposed 
project is user friendly or not, whether users can use the project 
without any training because of the user friendliness or not. Elearning System is very user friendly as its users interact with it 
through web.
\end{itemize}

\section{Software Requirement Analysis}
A Software Requirements Analysis for a software system is a complete 
description of the behaviour of a system to be developed. It includes 
a set of use cases that describe all the interactions the users will 
have with the software. In addition to use cases, the SRS also contains 
non-functional requirements. Non-functional requirements are 
requirements which impose constraints on the design or implementation.
\begin{itemize}
\item{\bf Purpose}: Elearning System is a web based software and the 
main purpose of this project is to:
\begin{enumerate}
\item Reduce the time spent in daily class room activities.
\item Make the Registration and Usage easier.
\item Automatically notifying the students about the upcoming events and tests.
\item Reduce the dependencies between people involved in the process.
\item Increasing the understanding between the teachers and students.
\item Keeping students up to date with all the work happening in class and also the future    events.
\end{enumerate}
\item{\bf General Description}: Elearning system is basically 
designed for those Organisations or Institutes which gives different 
types of work to all types of students. Keeping track of different 
works done by different students and then getting all the reports of 
the work done is not an easy job. To make these tasks easy with all 
functions performed quickly, Elearning system will be quiet helpful.

Administrator will be the super user of the application who will 
configure system information such as adding new students/teachers and their 
information or editing or deleting the old ones, managing students and teachers.

It will be an Institute software, so it is distributed and data centric. 
This Software is designed on the basis of web application architecture. 
In this application, MySQL database will be used to store data related 
to students, teachers, classes, events, institute, etc. Since 
database is on Server, so any number of users can work simultaneously 
and can share their data with each other. It is developed using PHP, HTML, CSS and J Query.
\item{\bf Users of the System}
\begin{enumerate} 
\item Administrator : Administrator can add or update 
(activate/inactivate) the details, and also can see information of all 
members of institute and can see his or her information. New classes, subjects and departments can be added or the existing can also be updated.
\item Teacher : As Teachers are directly related to students, so they 
are able to add or update the details of students using this section. 
Administrator can see all the students. Teachers can manage their 
students and class only, and particular student can see his or her detail.
\item Student : Students are the end users that benefit from the 
Elearning System. A student can get information of all services 
available. They can also view the upcoming events and also the important announcements
from the teacher of the institute.
\end{enumerate}
\end{itemize}
\subsection{Functional Requiremets}
\begin{itemize}
\item {\bf Specific Requirements}: This phase covers the whole requirements 
for the system. After understanding the system we need the input data 
to the system then we watch the output and determine whether the output 
from the system is according to our requirements or not. So what we have 
to input and then what we’ll get as output is given in this phase. This 
phase also describe the software and non-function requirements of the 
system.
\item {\bf Input Requirements of the System}
\begin{enumerate} 
\item Student Details
\item Teacher Details
\item Department Details
\item Class Details
\item Downloadable Material 
\end{enumerate}
\vskip 0.5cm
\item {\bf Output Requirements of the System}
\begin{enumerate} 
\item Interface for administrator to configure the system.
\item Listing of all the services offered.
\item Interface for students and teachers.
\item Generation of Results, Grades, Downloadable Material,
for students.
\item Calculation of student progress.
\item Generation of class calender and classmates list for student.
\item Generation of the list of classmates of a particular student.
\end{enumerate}
\vskip 0.5cm
\item {\bf Special User Requirements}
\begin{enumerate} 
\item Automatic Message Generation and Sending to the concerned person.
\end{enumerate}
\vskip 0.5cm
\item {\bf Software Requirements}
\begin{enumerate} 
\item Programming language: PHP 5.4
\item Web Languages: Html, J Query, CSS 
\item Database: MySQL Database Server 5.1 
\item Documentation: Doxygen 1.8.3
\item Text Editor: Gedit, Notepad++, Sublime
\item Operating System: Ubuntu 12.04 or up
\item Web Server: Apache 2.4
\end{enumerate}
\vskip 0.5cm
\subsection{Non functional requirements}
\begin{enumerate} 
\item Scalability: System should be able to handle a number of users. 
For e.g., handling around hundred users at the same time.
\item Usability: Simple user interfaces that a layman can understand.
\item Speed: Speed of the system should be responsive i.e. Response to
 a particular action should be available in short period of time. For 
e.g., Updating the class tasks take few seconds for the changes if 
the entry is not starred.
\end{enumerate}
\end{itemize}
