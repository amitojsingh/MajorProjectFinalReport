\section{Conclusion}
Automated Building Drawings System is a user friendly software system  which can be used to produce an automated drawing for any entity. One can easily understand the software and use it for their convenience. Anyone even a person having less knowledge of computers can use this software to crate a two dimesional drawings of many entities and also the objects. \\

\noindent Firstly the user input the required specifications of the desired design such as length, height, the type of entity etc. Then these parameters and all other details are saved in a input file which is a txt file. Then this file is parsed into the system where the processing is done and the input file is splitted into the output file where its parameters are separated into the lines.\\
 
\noindent As computer is used by nearly all the organizations so it is very easy to
use the softwares by the users as well as the officials and even the layman's. It would be very helpful to the people like layman and students,
or civil engineering people who dont have the time to draw the drawings in a paper and spending hours in correcting the designs. Thus it will help all types of people with their drawing process.
\section{Summary}
Automated Building Drawings System is used to eliminate the previous manual process of preparing the drawings manually by replacing it with the automated system, such that the end user can now easily use this software without any inconvenience. Every person who is directly or indirectly related to the the civil engineering work can use this software for their better working processes.\\

\noindent Before starting the project, the previous system is studied such that we get an idea how to proceed towards the project and keeping in mind its limitations new objectives have been set. We have gone through the Qt tutorials such that the software can be made easily and in time by revising some concepts of existing libraries. Then the various cad and drawing libraries are studied and we got an idea to proceed toward our project goals.\\
\noindent Firstly the user input the required specifications of the desired design such as length, height, the type of entity etc. Then these parameters and all other details are saved in a input file which is a txt file. Then this file is parsed into the system where the processing is done and the input file is splitted into the output file where its parameters are separated into the lines. After that the file is parsed into the system for the required output. The software processes the required parameters and produces an executable file. When that file is executed then we get a dxf file which is our required drawing output file. This file can be opened using any cad software which supports dxf format. We have used LibreCad in our project. 
 
\section{Future Scope}
This project can be used for drawing automation in a particular area where the
number of users are not well familiar with using the other softwares such as Freecad or Auto Cad. The drawing output records can be easily maintained by keeping the
data without using the pen paper work, thus increases productivity in the system by automating it. Output drawing file can be saved into other formats too like Pdf or the formats which can be directly open in some Cad software which will allow the modifications in the previous output of drawing thus it will lead to the reusability of the existing  designs, which will again increase the usefulness and efficiency of the manpower and it will also help in increasing the productivity.